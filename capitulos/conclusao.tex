\chapter{\textbf{Conclusão}} % Este comando é utilizado para criar capítulos

\section{Análise Geral do Trabalho}

O trabalho desenvolvido propiciou o desenvolvimento de um sistema de recomendação híbrido, prosseguindo com os estudos desenvolvidos nos trabalhos de \citeonline{tccluisa} e \citeonline{tcctiago}.

Com o desenvolvimento aberto para comunidade e disponibilização de uma API documentada, além de aplicações para visualização e geração das recomendações torna-se possível a utilização e melhoria do sistema por qualquer pessoa que possua interesse. Vale ressaltar que a estrutura do código foi desenvolvida buscando a modularização, o que acaba por possibilitar a adição de novas funcionalidades no sistema sem muito esforço.

A partir do estudo de caso aplicado no contexto educacional, foi observado que as filtragens híbridas obtiveram resultados estatisticamente superiores as filtragens colaborativa e baseada em conteúdo, quando utilizadas separadamente. Vale ressaltar o ótimo resultado obtido na abordagem híbrida mista que apresentou uma taxa de acerto cerca de 14\% superior as demais abordagens.

Em relação ao estudo de caso aplicado no contexto cultural, ainda foi observado resultados superiores na utilização das abordagens híbridas em relação à colaborativa e baseada em conteúdo. Nesse caso, em específico, é importante visualizar que os dados da abordagem híbrida mista divergem estatisticamente das avaliações dos usuários, fazendo com que os resultados obtidos na abordagem ponderada sejam mais confiáveis.

Acerca do uso da concorrência, infelizmente, não foi atingido o objetivo esperado, apresentando problemas no tempo de execução. Enquanto isso, a recomendação offline mostrou resultados satisfatórios podendo, desse modo, se apresentar como uma alternativa viável para melhoria de performance do sistema. 

\section{Trabalhos Futuros}

Como trabalhos futuros para sistema podem ser adicionados novas abordagens de recomendação, como a combinação sequencial e comutação, por exemplo, possibilitando o estudo e aplicação de novas técnicas. Novos sistemas de consulta e visualização de dados também podem ser desenvolvidos buscando atingir nichos ou plataformas diferentes.

Com o estudo mais profundo da área de arquitetura e desenvolvimento de sistemas distribuídos a técnica de concorrência pode apresentar-se como uma solução mais viável, podendo ser desenvolvida com sucesso em novos trabalhos.